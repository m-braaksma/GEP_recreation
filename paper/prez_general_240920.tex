% Options for packages loaded elsewhere
\PassOptionsToPackage{unicode}{hyperref}
\PassOptionsToPackage{hyphens}{url}
\PassOptionsToPackage{dvipsnames,svgnames,x11names}{xcolor}
%
\documentclass[
  letterpaper,
  DIV=11,
  numbers=noendperiod]{scrartcl}

\usepackage{amsmath,amssymb}
\usepackage{iftex}
\ifPDFTeX
  \usepackage[T1]{fontenc}
  \usepackage[utf8]{inputenc}
  \usepackage{textcomp} % provide euro and other symbols
\else % if luatex or xetex
  \usepackage{unicode-math}
  \defaultfontfeatures{Scale=MatchLowercase}
  \defaultfontfeatures[\rmfamily]{Ligatures=TeX,Scale=1}
\fi
\usepackage{lmodern}
\ifPDFTeX\else  
    % xetex/luatex font selection
\fi
% Use upquote if available, for straight quotes in verbatim environments
\IfFileExists{upquote.sty}{\usepackage{upquote}}{}
\IfFileExists{microtype.sty}{% use microtype if available
  \usepackage[]{microtype}
  \UseMicrotypeSet[protrusion]{basicmath} % disable protrusion for tt fonts
}{}
\makeatletter
\@ifundefined{KOMAClassName}{% if non-KOMA class
  \IfFileExists{parskip.sty}{%
    \usepackage{parskip}
  }{% else
    \setlength{\parindent}{0pt}
    \setlength{\parskip}{6pt plus 2pt minus 1pt}}
}{% if KOMA class
  \KOMAoptions{parskip=half}}
\makeatother
\usepackage{xcolor}
\setlength{\emergencystretch}{3em} % prevent overfull lines
\setcounter{secnumdepth}{5}
% Make \paragraph and \subparagraph free-standing
\ifx\paragraph\undefined\else
  \let\oldparagraph\paragraph
  \renewcommand{\paragraph}[1]{\oldparagraph{#1}\mbox{}}
\fi
\ifx\subparagraph\undefined\else
  \let\oldsubparagraph\subparagraph
  \renewcommand{\subparagraph}[1]{\oldsubparagraph{#1}\mbox{}}
\fi


\providecommand{\tightlist}{%
  \setlength{\itemsep}{0pt}\setlength{\parskip}{0pt}}\usepackage{longtable,booktabs,array}
\usepackage{calc} % for calculating minipage widths
% Correct order of tables after \paragraph or \subparagraph
\usepackage{etoolbox}
\makeatletter
\patchcmd\longtable{\par}{\if@noskipsec\mbox{}\fi\par}{}{}
\makeatother
% Allow footnotes in longtable head/foot
\IfFileExists{footnotehyper.sty}{\usepackage{footnotehyper}}{\usepackage{footnote}}
\makesavenoteenv{longtable}
\usepackage{graphicx}
\makeatletter
\def\maxwidth{\ifdim\Gin@nat@width>\linewidth\linewidth\else\Gin@nat@width\fi}
\def\maxheight{\ifdim\Gin@nat@height>\textheight\textheight\else\Gin@nat@height\fi}
\makeatother
% Scale images if necessary, so that they will not overflow the page
% margins by default, and it is still possible to overwrite the defaults
% using explicit options in \includegraphics[width, height, ...]{}
\setkeys{Gin}{width=\maxwidth,height=\maxheight,keepaspectratio}
% Set default figure placement to htbp
\makeatletter
\def\fps@figure{htbp}
\makeatother

\KOMAoption{captions}{tableheading}
\makeatletter
\makeatother
\makeatletter
\makeatother
\makeatletter
\@ifpackageloaded{caption}{}{\usepackage{caption}}
\AtBeginDocument{%
\ifdefined\contentsname
  \renewcommand*\contentsname{Table of contents}
\else
  \newcommand\contentsname{Table of contents}
\fi
\ifdefined\listfigurename
  \renewcommand*\listfigurename{List of Figures}
\else
  \newcommand\listfigurename{List of Figures}
\fi
\ifdefined\listtablename
  \renewcommand*\listtablename{List of Tables}
\else
  \newcommand\listtablename{List of Tables}
\fi
\ifdefined\figurename
  \renewcommand*\figurename{Figure}
\else
  \newcommand\figurename{Figure}
\fi
\ifdefined\tablename
  \renewcommand*\tablename{Table}
\else
  \newcommand\tablename{Table}
\fi
}
\@ifpackageloaded{float}{}{\usepackage{float}}
\floatstyle{ruled}
\@ifundefined{c@chapter}{\newfloat{codelisting}{h}{lop}}{\newfloat{codelisting}{h}{lop}[chapter]}
\floatname{codelisting}{Listing}
\newcommand*\listoflistings{\listof{codelisting}{List of Listings}}
\makeatother
\makeatletter
\@ifpackageloaded{caption}{}{\usepackage{caption}}
\@ifpackageloaded{subcaption}{}{\usepackage{subcaption}}
\makeatother
\makeatletter
\@ifpackageloaded{tcolorbox}{}{\usepackage[skins,breakable]{tcolorbox}}
\makeatother
\makeatletter
\@ifundefined{shadecolor}{\definecolor{shadecolor}{rgb}{.97, .97, .97}}
\makeatother
\makeatletter
\makeatother
\makeatletter
\makeatother
\ifLuaTeX
  \usepackage{selnolig}  % disable illegal ligatures
\fi
\IfFileExists{bookmark.sty}{\usepackage{bookmark}}{\usepackage{hyperref}}
\IfFileExists{xurl.sty}{\usepackage{xurl}}{} % add URL line breaks if available
\urlstyle{same} % disable monospaced font for URLs
\hypersetup{
  pdftitle={Ecosystem Service of Recreation for GEP},
  pdfauthor={Matthew Braaksma; Lifeng Ren; Spencer Wood; Ryan McWay},
  colorlinks=true,
  linkcolor={blue},
  filecolor={Maroon},
  citecolor={Blue},
  urlcolor={Blue},
  pdfcreator={LaTeX via pandoc}}

\title{Ecosystem Service of Recreation for GEP}
\author{Matthew Braaksma \and Lifeng Ren \and Spencer Wood \and Ryan
McWay}
\date{September 20, 2024}

\begin{document}
\maketitle
\ifdefined\Shaded\renewenvironment{Shaded}{\begin{tcolorbox}[frame hidden, boxrule=0pt, interior hidden, borderline west={3pt}{0pt}{shadecolor}, breakable, enhanced, sharp corners]}{\end{tcolorbox}}\fi

\renewcommand*\contentsname{Table of contents}
{
\hypersetup{linkcolor=}
\setcounter{tocdepth}{3}
\tableofcontents
}
\hypertarget{motivation}{%
\section{Motivation}\label{motivation}}

\begin{itemize}
\tightlist
\item
  We want to estimate the monetary value of recreation as an ecosystem
  service.
\item
  We would like to do this credibly at a global scale to produce a
  country-year panel.
\item
  This is a subcomponent of the larger GEP value for each nation and
  likely a sizeable value.
\end{itemize}

\hypertarget{components-of-recreation-gep}{%
\section{Components of Recreation
GEP}\label{components-of-recreation-gep}}

\begin{itemize}
\tightlist
\item
  Nature-based recreation can be sub-divided by the type of recreation
  (park, camping, beach, lake activites, eco-tourism, etc.).
\item
  The key distinction for nature-based recreation is that these are
  activities where the main attraction (or attribute deriving utility)
  comers from nature.
\item
  For our purposes, we can start by narrowing our examination to camping
  or hiking recreation in protected areas. But in general, we can modify
  the following approach to consider a wider definition of nature-based
  recreation.

  \begin{itemize}
  \tightlist
  \item
    Consider what are the \(i \in I\) that we wish to estimate and what
    we can't estimate.
  \end{itemize}
\end{itemize}

\hypertarget{previous-estimates-of-recreation-value}{%
\section{Previous Estimates of Recreation
Value}\label{previous-estimates-of-recreation-value}}

\begin{enumerate}
\def\labelenumi{\arabic{enumi}.}
\tightlist
\item
  SEEA Accounts (Vallecilloa et al.~2019, Ecological Modelling)

  \begin{itemize}
  \tightlist
  \item
    ESTIMAP-Recreation (EU Only)
  \item
    17\% of land suitable for daily recreation
  \item
    500 million inhabitants (62\% close to suitable land)
  \item
    Value: EUR 50 Billion
  \end{itemize}
\item
  GEP of Ecotourism in Qinghai (Ougang et al., 2020)

  \begin{itemize}
  \tightlist
  \item
    Contingent Valuation survey data of travel expenditures (P, Q) for
    `eco-tourism'
  \item
    They conduct the questionnaire at 3 parks of Qinghai and get the
    following key information

    \begin{itemize}
    \tightlist
    \item
      Visits Staying Time (days)
    \item
      Travel Expenditure (X \$/person)
    \item
      Salary of Visitor (Y \$/person/days)
    \item
      Where does the visitor come from (Zone Z)
    \end{itemize}
  \item
    Based on this information, they can get for all the zones they are
    interested in. And they get for each zone Z, what is the
    frequency/probability that people will visits 3 parks in Qinghai.
  \item
    Then they reweight by zones for all respondents' visiting rates and
    zonal population and sum up all costs (Travel Expenditure and Time
    expenditure) to equal to the GEP of ecotourism.
  \item
    462 respondents; 2 parks, 1 lake
  \item
    Value: Yuan 3 Billion in 2000, Yuan 21.6 Billion in 2015
  \end{itemize}
\end{enumerate}

\hypertarget{gep-formula}{%
\section{GEP Formula}\label{gep-formula}}

Deterministics (accounting) formula:

\begin{itemize}
\tightlist
\item
  Panel of monetary value over year \(t\) and country \(c\) over the set
  of recreation activities \(I\)
\item
  Requires aggreagtion to value by year and country to be consistent
  with system of national accounts (SNA) as presented with SEEA
  accounts.
\end{itemize}

\[
GEP_{c,t} = \lambda \cdot ( \sum_{i \in I} P_{i,c,t} \cdot Q_{i,c,t})
\]

\begin{itemize}
\tightlist
\item
  \(\lambda\): Nature's contribution
\item
  \(P\): Price \(\to\) Travel cost for park visitors
\item
  \(Q\): Quantity \(\to\) Number of park visitors
\end{itemize}

While \(\lambda\) may vary over time and space, there is a theoretical
and empirical decision to assert it is constant. Empirically, difficult
to measure a change over time. Theoretically, if the value of recreation
comes from contributions of land, labor, and capital -- is the relative
contribution of land compared to labor and capital changing
(meaningfully) over time? Does it vary over country (componsition of the
country)? Probably linked to structural transformation (development of a
country).

\hypertarget{estimating-recreation-gep}{%
\section{Estimating Recreation GEP}\label{estimating-recreation-gep}}

\hypertarget{method-1-extrapolation-from-gdp-measures}{%
\subsection{Method 1: Extrapolation from GDP
Measures}\label{method-1-extrapolation-from-gdp-measures}}

\hypertarget{general-model}{%
\subsubsection{General Model}\label{general-model}}

\[
GEP_{c,y} = \lambda \cdot \theta \times X_{c,y} \times Z_{c,y}
\]

\begin{itemize}
\tightlist
\item
  \(X\): Share of GDP from tourism
\item
  \(Z\): Annutal GDP
\item
  \(\lambda\): Nature's contribution adjustment
\item
  \(\theta\): \% of tourism that is nature based
\end{itemize}

\hypertarget{data-requirements}{%
\subsubsection{Data Requirements}\label{data-requirements}}

\begin{enumerate}
\def\labelenumi{\arabic{enumi}.}
\tightlist
\item
  Share GDP from tourism related to nature

  \begin{itemize}
  \tightlist
  \item
    \href{https://www.trade.gov/travel-and-tourism-satellite-account-ttsa-program}{Tourism
    Satellite Accounts (TSA)} for scenic and sight seeing travel cost
  \item
    \href{https://www.bea.gov/data/special-topics/outdoor-recreation}{BEA}
  \item
    \href{https://recreationroundtable.org/resources/national-recreation-data/}{ORR}
  \end{itemize}
\item
  National GDP measures

  \begin{itemize}
  \tightlist
  \item
    From the world bank
  \end{itemize}
\end{enumerate}

\hypertarget{calibration}{%
\subsubsection{Calibration}\label{calibration}}

\begin{itemize}
\tightlist
\item
  Test if this aligns with SEEA measures or UNWTO measures
\end{itemize}

\hypertarget{method-2-outdoor-recreation-extrapolation}{%
\subsection{Method 2: Outdoor Recreation
Extrapolation}\label{method-2-outdoor-recreation-extrapolation}}

\begin{itemize}
\tightlist
\item
  USA-ITA breaks down to GDP based on the recreation activities for both
  national level and state level
\item
  This data not available for the other countries (can extrapolate to
  other countries at global level) H- owever, it is hard to tease out
  the labor portion for calculating GDP\_recreation.
\end{itemize}

What we calculate now is an upper bound assuming

\[
GEP_{rec} \le GDP_{rec}
\]

\hypertarget{general-model-1}{%
\subsubsection{General Model}\label{general-model-1}}

Estimate the relationship (elasticity) between GDP recreation and size
of the recreation areas. Then if we know the recreation area size we can
use this as a proxy for GEP and translate this into monetary value
through GDP. The logic is that larger recreation areas will generate
more GDP.

Run the following state level regression

\[
GDP_{s,y} = \alpha + \beta X_{s,y} + \varepsilon
\]

\begin{itemize}
\tightlist
\item
  \(GDP\): State \(s\) by year \(y\) recreational GDP
\item
  \(X\): \% of recreational area divided by total size of the state
\item
  \(\beta\): elasticity between size of recreation area and GDP
  recreation share
\end{itemize}

With this estimate we can use the \(\beta\) to determine GEP for other
nations using the same approach

\[
GEP_{c,y} = \lambda \times (\beta \cdot X_{c,y})
\]

\begin{itemize}
\tightlist
\item
  \(X\): \% of protected area divided by total size of the country
\item
  \(\lambda\): Nature's contribution adjustment
\end{itemize}

A big limitation of this is that we are assuming that the relationship
of GDP to GEP for recreation in the US is representative of other
countries -- especially those in lower income brackets. This is unlikely
to be true.

\hypertarget{data-requirement}{%
\subsubsection{Data Requirement}\label{data-requirement}}

\begin{enumerate}
\def\labelenumi{\arabic{enumi}.}
\tightlist
\item
  Use USA Outdoor Recreation GDP at state level

  \begin{itemize}
  \tightlist
  \item
    Conventional recreation, boating/fishing, RVing, Snow activities
  \end{itemize}
\item
  Measure accessibility to state parks

  \begin{itemize}
  \tightlist
  \item
    \href{https://www.nature.com/articles/s41597-022-01857-7}{PAD-US-AR}
  \end{itemize}
\end{enumerate}

\hypertarget{method-3-balmford-et-al.-2015-measure}{%
\subsection{Method 3: Balmford et al.~2015
Measure}\label{method-3-balmford-et-al.-2015-measure}}

\hypertarget{estiamting-quantity-q}{%
\subsubsection{\texorpdfstring{Estiamting Quantity
\(Q\)}{Estiamting Quantity Q}}\label{estiamting-quantity-q}}

For each country \(c\), we can use the number of estimated visits
annually for each protected area using the
\href{https://journals.plos.org/plosbiology/article?id=10.1371/journal.pbio.1002074}{Balmford
et al.~2015} measure (\textasciitilde{} 8 Billion visits annually).

\[
Q_{c} = \sum_{i \in I} Visits_{ic}
\]

The main issue with this quantity it is likely can over-estimate of
visitations. It models predictions of visitation by:

\begin{itemize}
\tightlist
\item
  PA size
\item
  Local population size within 100KM of PA
\item
  Remoteness by travel time of city of \textgreater50,000 using global
  friction surface
\item
  Natural attractiveness using a 1-5 rating scale by 3 conservation
  scientists
\item
  National income
\end{itemize}

In their materials, they use a generalized linear model but (1) do not
show the estimation equation and (2) do not show validation of the model
or out of sample prediction.

\hypertarget{estimating-price-p}{%
\subsubsection{\texorpdfstring{Estimating Price
\(P\)}{Estimating Price P}}\label{estimating-price-p}}

Try to estimate the average expenditure for a recreation visit by
dividing the GDP per capita attributable to recreation by the number of
park visitors annually. You could measure the expenditure on recreation
through
\href{https://ourworldindata.org/grapher/tourism-gdp-proportion-of-total-gdp?time=2022}{UNWTO
share of GDP} or as
\href{https://data.worldbank.org/indicator/ST.INT.RCPT.CD?end=2020\&start=1995}{international
tourism reciept} (a bit more broadly defined).

Two issue:

\begin{itemize}
\tightlist
\item
  GDP measure will be for all recreation (includes tourism, etc.). So
  this will over-value the price
\item
  It is difficult to find data on only nature-based recreation GDP
\end{itemize}

\[
P_c = \frac{GDP_{pc} from Rec_j}{Q_c}
\]

\hypertarget{method-4-ryans-imagination}{%
\subsection{Method 4: Ryan's
Imagination}\label{method-4-ryans-imagination}}

\hypertarget{estimating-quantity-q}{%
\subsubsection{\texorpdfstring{Estimating Quantity
\(Q\)}{Estimating Quantity Q}}\label{estimating-quantity-q}}

\begin{itemize}
\tightlist
\item
  Main issue: We do not observe park vistors in systematic way across
  global and across time.
\end{itemize}

Inspired by the
\href{https://journals.plos.org/plosbiology/article?id=10.1371/journal.pbio.1002074}{Balmford
et al., 2015} method. Ideally improved. What I am aiming to do is
measure a propensity to visit using attributes of the park that are
attractive and the distance of the park. Then multiplying this
probability to visit by the known population from the origin to obtain
the number of visitors a park gets each year.

Start with the
\href{https://sedac.ciesin.columbia.edu/data/collection/gpw-v4}{gridded
population data}. The idea is to estimate a likelihood function to
determine the probability that a representive agent in any grid cell
will visit each park. We can estimate this primarly using distance to
parks. In this setting, we assume that likelihood to visit is primarly
determined by distance. But we add in other attributes that make a park
attractive (just as is done in Balmford).

The measure of distance can be determined by:

\begin{itemize}
\tightlist
\item
  Euclidean distance using raster algebra
\item
  Travel time calculated by Open Street Maps
\end{itemize}

We can place this framework within a gravity model.

\[
V_{ij} = G \frac{M_i^{\beta_1}M_j^{\beta_2}}{D_{ij}exp(Border)} \cdot \eta
\]

\begin{itemize}
\tightlist
\item
  \(G\): Base rate of gravity. In our context this is the base rate of
  visitiation between grid cell \(i\) and park \(j\). This would be the
  frictionless rate of visitation (perfectly accessible park)
\item
  \(M_i\) and \(M_j\): These are the attration properties of grid cell
  \(i\) to park \(j\). For grid cells, this may be that they are
  urban/rural, income, or other attributes that make them have increased
  desire to visit a park. For parks, this could be attributes about the
  park like features, scenic value, size of the park, designation as
  national, state or local.
\item
  \(D\): the distance between grid cell \(i\) and park \(j\)
\item
  \(Border\): This is dummy variable that applifies the fricition to
  distance if the visitor has to cross a national border (administrative
  or logistical fricitions)
\item
  \(\eta\): A multipilicative error term for all other attributes that
  pull the bodies together and push the apart. Otherwise, this is a
  deterministic model.
\end{itemize}

We can log-linearize this model:

\[
log(V_{ij}) = \beta_0 log(G) + \beta_1 log(M_i) + \beta_2 log(M_j) + \beta_3 log(D_{ij}) + \beta_4 Border + \eta
\]

We will need to match this model to real data. So to take this to the
data we will likely want to use a Poisson Pseudo Maximum Likelihood
(PPML) method that accounts for heteroskedastic errors (Silva and
Tenreyro, 2006). We will need some data on the number of visits from
\(i\) to \(j\). So this is some park level information on annual visits
generally from across the world. This is difficult to do for all parks
globally. But suppose you can do it for some subset of parks that are
representative of all parks (externally valid). Then you can estimate
the above model and use the \(\beta\) coefficents as a calibration
method to extrapolate the number of visitations \(V\) for each park
given attributes that are known globally about \(M_i\) and \(M_j\) and
our estimated distance between the two locations.

At the end of this exercise, we will have the number of estimated visits
from each park to each grid cell. This allows us to generate:

\begin{enumerate}
\def\labelenumi{\arabic{enumi}.}
\tightlist
\item
  The total number of visitations to each park.
\item
  The total visitations from each grid cell.
\end{enumerate}

Ideally, we can validate our results for each with some aggregated data
on park vistations from tourism and park services for parks and from
cities. This should add to the credibility of our estimates for the
quantity of visitors.

In the end, for GEP we will just need to aggregate up the number of
visits by \emph{parks} to get the \(Q\) in the forumla for each country.
We should use park value (rather than grid values) because that is where
the ecosystem service is being derived from. So we don't actually care
where the visitor comes from. What we care about is where they go. That
is where the production value is being produced (and quinsendentally
consumed as this is a non-tradeable good).

A Suggestion by Steve: Do a zonal calculation. Only consider people
within a certian proximity of each park rather than the whole world.
This relies on the assumption that after a certain buffer, travel to a
PA is prohibitive (either in expense or legally).

\hypertarget{estimating-price-p-1}{%
\subsubsection{\texorpdfstring{Estimating Price
\(P\)}{Estimating Price P}}\label{estimating-price-p-1}}

\begin{itemize}
\tightlist
\item
  Main issues: (1) We do not observe prices for park visitors, and (2)
  most recreation areas are often free access or subsidized so we have a
  non-market good to consider.
\end{itemize}

The underlying principle to estimate the price for recreation is to
measure a Willingness to Pay (WTP) through revealed preference of travel
cost. The underlying assumption in this setting is that the primary
determinate of visiting a park is distance. So the further the distance,
the more valuable the park must be to the visitor. Note that this can be
confirmed from the gravity model in the previous section using
\(\beta_3\). This should be the largest coefficient relative to
observable attributes in \(M_i\) and \(M_j\). If it is not, then
distance is not the main driver of revealed preferences.

A nice attribute of travel cost is that we can associate this cost to
known costs of travel in monetary terms. So, persume (naively) that
everyone drives or flies to get to their destination. Then we can use
gasoline prices as a measure of cost and multiply that by the distance
traveled. You can imagine how this could be refined. In this way, we
avoid the infeasible necessity to use a survey method for visitors to
report costs. This can be a way to validate our cost measures if there
are parks/studies with this information. Such an exercise would be
designed to change the travel costs to better reflected survey reports
to total cost. So you could imagine the following simple regression to
go beyond a basic gas prices \(\times\) distance measure.

\[
C_{ij} = \alpha_0 + \alpha_1 G_{i} \times D_{ij} + \alpha_2 X_{i} + \varepsilon 
\]

\begin{itemize}
\tightlist
\item
  \(C\): Cost paid by visitor \(i\) to go to \(j\)
\item
  \(G\): Gas prices paid by visitor \(i\)
\item
  \(D\): Distance for visitor \(i\) to \(j\)
\item
  \(X\): Any other costs that are reported
\end{itemize}

From this regression, what can use from the subset of survey data is
determine the influence of gas and distance \(\alpha_1\) relative to the
importance of all other costs \(\alpha_2\). We can also confirm in this
way that gas costs from distance is the main travel cost. We know the
distance and gas globally, but not the other information. So we can try
to measure the porportional importance of other costs to total cost. Say
something as simple as \(\frac{D}{C}\) as a probability weight for
portion of cost attributable to distance. Flip this an we can re-weight
all of our travel costs with an inverse probability weight. Namely,
\(w = \frac{1}{\frac{D}{C}} = \frac{C}{D}\).

From here forward I am laying out the groundwork for backing out the WTP
measure in monetary terms. In case it gets lost in my explanation, the
core aspect we care about is likelihood to visit determined by
importance of income net of travel costs. This should give us the price
for any given park as determined by distance.

Let us start with a discrete choice model. In this model, visitor \(n\)
is selecting which park \(j\) they would like to visit. In any time
\(t\) they can only select one park \(j\) from the set of parks \(J\).
We observe ex-post where a person has choosen to visit. So we will
develop a behavioral model to describe why they made that choice using
observable attributes about the individual visitor and the selection of
parks avaliable to them.

Let us start with a random utility model (REM). The visitor has observed
attributes \(s_n\). We will say that these are heterogenous preferences
across the population. So if we observe differences in income or being a
foreign national we can distinguish visitation by differences in tastes.
The visitory selects over alternative parks \(j\) which have observable
attributes \(x_{nj} \forall j \in J\). These are things like scenic
quality, size of the park, amenities, travel cost to get there, etc.
Each visitor recieves some representative utlity for each park they
\emph{could} visit:

\[
V_{nj} = V(x_{nj}, s_n) = \beta'_n x_{nj} \forall j
\]

\(\beta'_n\) allows for variation in taste and this is described by an
i.i.d. extreme value distribution \(\beta_n \sim f_n(\beta|\Theta)\).
The \(\beta\) captures the \(s_n\) attributes varying across vistors.
This is their difference in assigning importance (perference) over
different attributes of the park \(x\). This is non-deterministic as
there is utility from attributes we don't observe. So this is an
indirect utility function. It is a component of the following overall
direct utility function:

\[
U_{nj} =  V_{nj} + \varepsilon_{nj}
\]

Let us assume that the unobserved attributes as a random vector
logistically distributed by \(f(\varepsilon)\). The probability that
visitor \(n\) visits park \(i\) is

\[
P_{ni} = \int \mathbf{1}[\varepsilon_{nj} - \varepsilon_{ni} <  V_{ni} -  V_{nj}] f(\varepsilon) d \varepsilon
\]

This simply states that the probability of vistor \(n\) selecting park
\(i\) to visit is the value obtained from observed attributes is greater
in \(i\) than in all other parks \(j\) and that this value is greater
than the unobserved value for park \(j\) over park \(i\). Note also that
this is a CDF.

We assume that \(\varepsilon\) is i.i.d. from the extreme value
distribution such that:

\[
CDF: F(\varepsilon_{nj}) = 1 - e^{-e^{\varepsilon_{nj}}}
\]

Then the realized difference that the random error value for \(j\) is
greater than \(i\) becomes:

\[
\begin{align}
\varepsilon^*_{nji} = & \varepsilon_{nj} - \varepsilon_{ni} \\ 
F(\varepsilon^*_{nji}) = & \frac{e^{\varepsilon^*_{nji}}}{1 + \varepsilon^*_{nji}} 
\end{align}
\]

From here we get the probability of visitor \(n\) visiting park \(i\)
where \(V_{nj} = \beta'_n x_{nj}\) as:

\[
P_{ni} = \int \left( \frac{e^{\beta'_n x_{ni}}}{\sum_j e^{ \beta'_n x_{nj}}} \right) f_n(\beta|\Theta) d(\beta)
\]

Because we are allowing for taste variation in the population, the above
proability does not have a closed form solution (because we do not know
the solution to the mixing of several logistical distributions -- one
\(\varepsilon\) distribution for each \(\beta\)). But we can give the
probability distrubition some normal form:
\(f_n(\beta|\Theta) = f(\beta|\mu, \sigma)\). And an advantage here is
that we have relaxed the Irrelevance of Independent Alternatives (IIA)
assumption - we have correlations between unobserved factors of
alternatives.

We will construct a simulated mixed logit proability through
bootstrapping and then place this within a simulated log-likelihood
estimator. For the simulated mixed logit, we will do the following
steps:

\begin{enumerate}
\def\labelenumi{\arabic{enumi}.}
\tightlist
\item
  Draw at \(\beta^r\) for draw \(r\) from \(f_n(\beta|\Theta)\). This is
  an observed attribute for visitor \(n \in N\).
\item
  Calculate the logit formula for \(P_ni(B^r)\). This is the probability
  the visit park \(i\) given we observe this attribute about the visitor
  and all the attributes about the park \(x_j \forall j \in J\).
\end{enumerate}

\[
P_ni(B^r) = \frac{e^{B^r x_{ni}}}{\sum_j e^{B^r x_{nj}}}
\]

\begin{enumerate}
\def\labelenumi{\arabic{enumi}.}
\setcounter{enumi}{2}
\tightlist
\item
  Re-draw and estimate 1 and 2 \(R\) times
\item
  Average the results to get simulated probability
\end{enumerate}

\[
\hat{P_{ni}} = \frac{1}{R} \sum_{r=1}^R P_ni(B^r)
\]

Now we can place it in the simulated log-likelihood estimator. Set
\(y_{ni} = 1\) when the visitor chooses park \(i\). We select the
\(\Theta\) that matches \((\mu, \sigma)\) to the observed data using
general method of moments. This is the estimator:

\[
SSL = \sum_{n=1}^N \sum_{j=1}^J y_{nj} log(\hat{P_{ni}})
\]

In our context, to back out the monetary value of travel cost we need
the following indirect utility function that we can vary over time
\(t\):

\[
V_{njt} = \beta_1 (y_n - c_{njt}) + \varepsilon_{njt}
\]

\(\beta_1\) is the WTP price we are after for visitor \(n\) to park
\(j\) in time \(t\). In our case, is the likelihood that a visitor from
grid cell \(n\). We can add in other attributes we think are correlated
with \(\beta_1\) that may bias it. But in practice, all we need to know
is average income \(y\) and some measure of travel cost \(c\).

\hypertarget{adjusting-for-natures-contribution-lambda}{%
\subsubsection{\texorpdfstring{Adjusting for Nature's Contribution
\(\lambda\)}{Adjusting for Nature's Contribution \textbackslash lambda}}\label{adjusting-for-natures-contribution-lambda}}

Two potential Methods:

\begin{enumerate}
\def\labelenumi{\arabic{enumi}.}
\tightlist
\item
  Something out of GTAP or CWON (both by Justin).

  \begin{itemize}
  \tightlist
  \item
    From cobb-douglas production function, \(1 - \alpha - \beta\).
  \end{itemize}
\item
  Try a hedonic model for land prices

  \begin{itemize}
  \tightlist
  \item
    Value of park land with no development must be the value of nature
  \item
    Nothing else can contribute to the value of land in protected area
  \item
    Would need to make this a multiplier effect (so portion of the
    hedonic value attributable to land)
  \item
    Issue: Because it is protected land, we do not see land purchases
    very often. In fact, most valuable land is likely always a protected
    area.
  \end{itemize}
\end{enumerate}

\hypertarget{data-requirements-1}{%
\subsubsection{Data Requirements}\label{data-requirements-1}}

\begin{enumerate}
\def\labelenumi{\arabic{enumi}.}
\tightlist
\item
  World Protected Areas Database

  \begin{itemize}
  \tightlist
  \item
    Shapefiles for location and size of existing protected areas and
    when they open (allows for dyanmic component)
  \end{itemize}
\item
  InVEST Scenic Quality Model

  \begin{itemize}
  \tightlist
  \item
    https://naturalcapitalproject.stanford.edu/invest/scenic-quality
  \item
    Determine avg (or max) attractiveness of each park
  \item
    Need info on viewpoints, visibility of view points (DEM), and weight
    for value of each view point (relative value)
  \end{itemize}
\item
  Gridded Population Data

  \begin{itemize}
  \tightlist
  \item
    \# of people in each grid cell and where they live.
  \end{itemize}
\item
  Incomes at sub-national level

  \begin{itemize}
  \tightlist
  \item
    Assign as the average income (budget constraint) for people in given
    grid cells
  \end{itemize}
\item
  Open Street Map

  \begin{itemize}
  \tightlist
  \item
    Average travel distance and time from grid cells to parks
  \end{itemize}
\item
  Park Visititation Data

  \begin{itemize}
  \tightlist
  \item
    Natual Park Service, etc.
  \item
    Visitation of rates to calibrate visitation model off of.
  \end{itemize}
\end{enumerate}

\hypertarget{analysis}{%
\subsubsection{Analysis}\label{analysis}}

\hypertarget{cleaning}{%
\paragraph{Cleaning}\label{cleaning}}

\begin{enumerate}
\def\labelenumi{\arabic{enumi}.}
\tightlist
\item
  Clean raw datasets
\item
  Merge attributes of parks with park level info
\item
  Merge population gridded data with attributes about consumers (income)
\end{enumerate}

\hypertarget{estimate-grid-cell-components}{%
\paragraph{Estimate Grid Cell
Components}\label{estimate-grid-cell-components}}

\begin{enumerate}
\def\labelenumi{\arabic{enumi}.}
\setcounter{enumi}{3}
\tightlist
\item
  Estimate distance from gridded cells to parks (Euclidean distance, or
  OSM distance)
\item
  Estimate propensity for visits by grid cells
\item
  Calibrate propensity for visits = quantity of visitors per park
\item
  Estimate WTP for visitors in each grid cell to each park
\item
  Back out price using WTP, income, and travel cost = price for visitors
  for park-grid cell pairs
\end{enumerate}

\hypertarget{estimate-gep}{%
\paragraph{Estimate GEP}\label{estimate-gep}}

\begin{enumerate}
\def\labelenumi{\arabic{enumi}.}
\tightlist
\item
  Multiply grid cell price-to-park with grid cell visitor-to-park
  \(\times\) nature's contribution adjustment = GEP value to visit all
  parks from that grid cell
\item
  Aggregate grid cell value for a nation in a year = \(GEP_{i,t}\) =
  Export panel data
\end{enumerate}

\hypertarget{output}{%
\section{Output}\label{output}}

\begin{enumerate}
\def\labelenumi{\arabic{enumi}.}
\tightlist
\item
  A panel data set of GEP values for country and year we can hand to
  Justin \& Steve
\item
  Map showing the gridded population overlayed on protected areas
  (Method)
\item
  Map (or table) showing grid cell values for total visitor count and
  average WTP (Origin)
\item
  Map (or table) showing parks total visitor count and average WTP
  (Destination)
\item
  Time series of GEP value
\item
  Some comparision (ground truth table) of recreation GEP value to GDP
  value (World bank measure)

  \begin{itemize}
  \tightlist
  \item
    https://www.unwto.org/tourism-statistics/economic-contribution-SDG
  \item
    https://wdi.worldbank.org/table/6.14
  \item
    https://data.worldbank.org/indicator/ST.INT.RCPT.CD
  \item
    https://data.oecd.org/industry/tourism-gdp.htm
  \end{itemize}
\end{enumerate}



\end{document}
